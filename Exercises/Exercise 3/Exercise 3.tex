% -------------------------------------- PREAMBLE STARTS HERE --------------------------------------------
% This is the preamble. It sets the main options for your document and load the packages used.

% Load packages: making changes here can cause errors.
\documentclass{article}            % Define document class. Shouldn't change.   
\usepackage[margin=1in]{geometry}
\usepackage{adjustbox}                  % This package allows you to adapt table and figure sizes to fit the page and is required by iebaltab
\usepackage{float}
\usepackage{setspace} 
\usepackage{hyperref}      
\onehalfspacing             % Uncomment to use double spacing

% ADD YOUR PROJECT INFO HERE 
\title{Exercise 3} 	% Double backslash starts a new line in \LaTeX
\author{Kristoffer Bjarkefur, Luiza Andrade \& Mrijan Rimal}
\date{\today}                    							% Prints date when compiling


% -------------------------------------- PREAMBLE ENDS HERE --------------------------------------------


\begin{document}           

\maketitle
\newpage

\section*{{\LaTeX} environments}
Environments in {\LaTeX} are blocks of code which have special formatting rules. In a {\LaTeX} document, environments are defined by \verb|\begin{}| and an \verb|\end{}| tag and everything in those \textit{environment} is formatted in the manner predefined by the environment. For example - between the \verb|\begin{center}| and \verb|\end{center}|, content(tables, figures, text) is centered as the formatting rule for the center environment is used. Similarly for content in a \texttt{table} environment i.e. between \verb|\begin{table}| and \verb|\end{table}| is formatted as a table as predefined in the table environment. 

Environments can be anything ranging from very something simple like using a certain text justification(i.e. center, left, right justified) to something more complex like math mode used in typing equations, tables, or figures. 

Some of the more common {\LaTeX} environments we have seen in Exercise 1 and 2 are listed below with a short explanation of what they do: 

\begin{description}
	\item[figure] This environment is used to import figures and graphs into {\LaTeX}. 
	\item[table] This environment is used to create, or import a table into a {\LaTeX} document.
	\item[adjustbox] This environment helps in adjusting the tables we created to fit the page we are working on. This environment is also generally used inside the \texttt{table} or \texttt{figure} environments.
	\item[enumerate] The \texttt{enumerate} environment is used to created numbered lists in {\LaTeX}. 
	\item[itemize] The \texttt{itemize} environment is used to created bullet point lists in {\LaTeX}.
\end{description} 

You can even create your own environments which you can use in your document. However, this will not be covered in this exercise. If you would like to learn more about creating your own {\LaTeX} environment, please go to the following \url{https://en.wikibooks.org/wiki/LaTeX/Macros#New_environments}.

\section*{Typing math symbols in {\LaTeX}}
While writing papers in {\LaTeX}, there might be a need to type up math symbols and equations. Math equations and symbols can be typed using the ma
\end{document}     