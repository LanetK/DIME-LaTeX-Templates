% -------------------------------------- PREAMBLE STARTS HERE --------------------------------------------
% This is the preamble. It sets the main options for your document and load the packages used.

% Load packages: making changes here can cause errors.
\documentclass{article}                 % Define document class. Shouldn't change.
\usepackage{booktabs}       
\usepackage{tabularx}   
\usepackage{import}                     % This package allows us to import files. 
\usepackage{multirow}
\usepackage{adjustbox}                  % This package allows you to adapt table and figure sizes to fit the page and is required by iebaltab
\usepackage{geometry}
\usepackage{subcaption}                 % This packages is used to create subfigures
\usepackage{float}

\usepackage{setspace}       
\doublespacing                          % Uncomment to use double spacing
\usepackage{indentfirst}	            % Indents the fist paragraph of each section
\usepackage{parskip}                    % This packages sets the spacing between two paragraphs
\setlength{\parskip}{.5\baselineskip}   % Define spacing between two paragraphs


\usepackage{hyperref}
% ADD YOUR PROJECT INFO HERE 
\title{DIME template for \LaTeX \\ Importing tables} 	% Double backslash starts a new line in \LaTeX
\author{Luiza Andrade \& Mrijan Rimal}
%\date{}                    							% Uncomment this to not print date or insert specific date

% -------------------------------------- PREAMBLE ENDS HERE --------------------------------------------


\begin{document}            % Will not compile without this line

    \maketitle
    \tableofcontents        % Comment out to not print summary
    \listoftables			% Comment out to not print the list of tables

    \newpage    
    \section{Introduction} %The titles are created automatically after using this command. This command is also useful as you can create a table of content just by using this. There is no need to do anything extra. 
    
        This template was created to make your life easier. You should be able to create a PDF file with the tables you generated by just editing the path to the files you exported from Stata.
     
    \section{Tables}

   		% To import a table created by iebaltab, just edit the path below to match the location of your file. Remember all file paths must start in the same folder as the main .tex file is (relative file path)
   		\input{../Raw/balance_table.tex}
           	   
         % To import a fragmented tex file -- as created by esttab, for example -- just edit the path after \input
		\begin{table}[H]
			\caption{Add table title}
			\input{../Raw/categorical.tex}
		\end{table}
           
  
\end{document}      % Will not compile without this line